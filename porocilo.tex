\documentclass[12pt, a4paper]{article}
\usepackage{graphicx}
\usepackage[slovenian]{babel}
\usepackage{csquotes}
\usepackage{biblatex}
\usepackage{amsmath}
\addbibresource{porocilo.bib}
\graphicspath{{images/}}
\title{Zharko - upodabljanje s sledenjem žarkov}
\author{Filip Trplan}
\date{Avgust 2025}

\begin{document}
\maketitle

\begin{figure}[h]
	\centering
	\includegraphics[width=\textwidth]{cover}
	\caption{Koncni rezultat}
\end{figure}

V temu porocilu bom predstavil tehniko upodabljaljanja (\textit{rendering}) s sledenjem zarkov
(\textit{ray tracing}). Lotil se bom prvo osnov sledenja zarkov in bom nato presel na ustvarjanje treh
razlicnih tipov materialov ter koncal z simulacijo globinske ostrine (\textit{depth of field}).

Vseskozi porocila se zgledujem po knjigi \textit{Ray Tracing in One Weekend} \cite{Shirley2025RTW1}

\section{Osnove sledenja zarkov}

Osnovna ideja upodabljanja s sledenjem zarkov je, da si predstavljamo prikazano sliko kot ploskev in kamero
orientirano v 3D prostoru. V realnem svetu svetloba izvira iz svetlobnih virov kot so sonce ali luci in se
nato odbije od raznih objektov dokler ne doseze nasega ocesa oz. senzorja na kameri. Ker pa veliko teh zarkov
zgresi naso kamero, bomo obrnili smer zarkov in jih bomo izstrelili iz kamere.

Zarek bo potoval iz kamere skozi piksel nasem vidnem oknu (\textit{viewport}) ter se nato odbil po
virtualnem svetu dokler ne bo dosegel svetlobnega vira (v nasem primeru bo to nebo).

\begin{figure}[h]

	\caption{Shema sledenja zarkov}
\end{figure}

Vpeljimo par oznak, da si bomo lazje predstavljali kako te zarki potekajo. Pozicijo kamere bomo imenovali $C$,
levi zgornji piksel bo $P_{00}$ in ce so piksli na mrezi, bodo enakomerno razporejeni tako da bo v horizontalni
smeri jih locil vektor $\Delta u$ in v vertikalni $\Delta v$. Prikazane so na sliki \ref{fig:oznake}.

\begin{figure}[h]
	\centering
	\includegraphics[width=\textwidth]{vidno_okno}
	\caption{Oznake prikazane vizualno}
	\label{fig:oznake}
\end{figure}

Potem lahko zarek $\vec{r}$, ki potuje skozi piksel $(i,j)$ parametriziramo s $t$:

\begin{equation}
	\vec{r} = C + t \cdot (P_{00} + i \cdot \Delta u + j \cdot \Delta v - C)
\end{equation}

V prihodnje pa bomo obravnavali splosne zarke z izvorom v $O$ in smerjo $\vec{d}$.

\section{Detekcija trka}

Trenutno nasi zarki samo streljajo v praznino zato bomo zdaj v svet dodali objekte v katere lahko zarki trcijo.
Matematicno je najlazje opisati sfero. Splosna enacba za sfero s srediscem $C$ in radijem $r$ je

\begin{equation}
	\label{eq:sfera}
	(C_{x} - x)^2  + (C_{y} - y)^2  + (C_{z} - z)^2  = r^2
\end{equation}

Ce tocko $(x,y,z)$ oznacimo kot $P$ lahko \ref{eq:sfera} zapisemo kot skalarni produkt.

\begin{equation}
	(C - P)(C - P) = r^2
\end{equation}

Nasa tocka pa lezi na zarku torej dobimo

\begin{equation}
	(C - (O + t \vec{d}))(C - (O + t \vec{d})) = r^2
\end{equation}

Po nekaj preurejanja enacbe dobimo naslednjo kvadratno enacbo

\begin{equation}
	t^2 \lVert C \rVert ^2  - 2 t \cdot \vec{d} (C - O) + \lVert C - O \rVert ^2  - r^2 = 0
\end{equation}

S pomocjo diskriminante lahko zlahka preverimo ali nas zarek sploh zadane sfero. Ce resitev obstaja pa vzamemo
najmanjso pozitivno (torej najblizjo kameri v smeri pogleda). V primeru, da pozitivna resitev ne obstaja pa
smatramo, kot da zarek ni zadel objekta, saj je za kamero.

Na sliki \ref{fig:red_sphere} vidimo primer izrisa sfere, kjer jo samo obarvamo samo z rdeco barvo.

\begin{figure}[h]
	\label{fig:red_sphere}
	\includegraphics[width=\textwidth]{red_sphere}
	\caption{Rdece obarvana sfera}
\end{figure}

Na zgornji sliki vidimo, da ima sfera zagast rob. To se zgodi zaradi diskretizacije pikslov v mrezo. Temu
pojavimo pravimo \textit{aliasing}. Znebimo se ga tako, da za vsak piksel izstrelimo vec zarkov ki so nakljucno
pertrubirani za najvec polovicno razdaljo med piksli, ter nato barvo teh zarkov povprecimo. Tehnika se imenuje
MSAA (\textit{multisample anti-aliasing}) in smo jo izbrali, ker je v kontekstu sledenja zarkov najbolj
enostavna za implementirati.

\section{Materiali}

\subsection{Difuzni materiali in Lambertov odboj}

Prve materiali, ki se jih bomo lotili so difuzni. Taki materiali v realnem svetu zgledajo mat in so takega
videza, ker se zarki nakljucno odbijejo od povrsine objekta. Lambertov odboj je aproksimacija takih materialov,
ki pravi, da naj je smer odbitega zarka izbrana sorazmerno s kolicino $\cos (\phi) $, kjer je $\phi$ vpadni
kot.

\begin{figure}[h]
	\centering
	\includegraphics[height=250pt]{shema_difzuni}
	\caption{Lambertov odboj}
\end{figure}

Ce zarek seka objekt na tocki $P$ in je normala povrsine vektor $n$, potem izberemo nakljucno tocko $Q$
znotraj enotske sfere s srediscem $P + n$ (kjer je normala enotski vektor). Smer odbitega zarka je potem
$Q - P$.

\begin{figure}[h]
	\centering
	\includegraphics[width=400pt]{difuzni}
	\caption{Dve difuzni sferi}
\end{figure}

Da dodamo barvo, odbiti žarek pomnožimo z barvo objekta, ki ji pravimo \textit{albedo}. Tako se barva
svetlobnega vira postopoma spreminja ob vsakem odboju - množi se z albedo vsakega objekta, s katerim se
žarek sreča. Končna barva piksla je zato produkt barve svetlobnega vira in vseh albedo vrednosti
objektov na poti žarka.

\textit{Opomba: }, ker se ta metoda zanasa na nakljucne procese za aproksimacijo realnega sveta (Monte Carlo),
dobimo nekaj suma v sliki.

\subsection{Kovinski materiali}

Kovine imajo svojo barvo, vendar pa hkrati tudi odbijejo del svetlobe pod istim kotom, kot je
zarek srecal objekt. Odbiti zarek je sestavljen iz dveh delov: zrcaljeni del in t.i. zamegljeni del
(\textit{fuzziness}). Koncni odbiti zarek je vsota teh dveh vektorjev.

\begin{figure}[h]
	\centering
	\includegraphics[width=400pt]{metal}
	\caption{Odboj zarka od kovinskega materiala}
\end{figure}

Zrcaljeni del je preprosto vpadni zarek zrcaljen cez ploskev, ki jo opise normala in z zamenjano smerjo.
Zameglejni del pa simulira hrapavost kovine in je samo nakljucni vektor znotraj enotske sfere pomnozen
s parametrom $0 \leq f \leq 1$, ki kontrolira kako zamegljen je objekt.

\printbibliography

\end{document}
