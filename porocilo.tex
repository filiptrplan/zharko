\documentclass[12pt, a4paper]{article}
\usepackage{graphicx}
\usepackage[slovenian]{babel}
\usepackage{csquotes}
\usepackage{biblatex}
\usepackage{amsmath}
\usepackage{float}
\addbibresource{porocilo.bib}
\graphicspath{{images/}}
\title{Žarko - upodabljanje s sledenjem žarkov}
\author{Filip Trplan}
\date{Avgust 2025}

\begin{document}
\maketitle

\begin{figure}[h]
	\centering
	\includegraphics[width=\textwidth]{cover}
	\caption{Končni rezultat}
\end{figure}

V tem poročilu bom predstavil tehniko upodabljaljanja (\textit{rendering}) s sledenjem žarkov
(\textit{ray tracing}). Prvo se bom lotil osnov sledenja žarkov, nato prešel na ustvarjanje treh
različnih tipov materialov, ter končal z simulacijo globinske ostrine (\textit{depth of field}).

Tekom poročila se zgledujem po knjigi \textit{Ray Tracing in One Weekend} \cite{Shirley2025RTW1}

\section{Osnove sledenja žarkov}

Osnovna ideja upodabljanja s sledenjem žarkov je, da si predstavljamo prikazano sliko kot ploskev in kamero
kot točko v 3D prostoru. V realnem svetu svetloba izvira iz svetlobnih virov kot so sonce ali luči,
in se nato odbije od raznih objektov, dokler ne doseže našega očesa oz. senzorja na kameri.
Ker pa veliko teh žarkov zgreši kamero, bomo obrnili njihovo smer in jih bomo izstrelili iz kamere.

Žarek bo potoval iz kamere skozi piksel na našem vidnem oknu (\textit{viewport}) ter se nato odbil po
virtualnem svetu dokler ne bo dosegel svetlobnega vira (v našem primeru bo to nebo).

\begin{figure}[H]
	\includegraphics[width=\textwidth]{shema_zarki}
	\caption{Shema sledenja žarkov}
\end{figure}

Vpeljimo par oznak, da si bomo lažje predstavljali kako te žarki potekajo. Pozicijo kamere bomo imenovali $C$,
levi zgornji piksel bo $P_{00}$. Piksle bomo enakomerno razporedili v mrežo, tako da jih bo v horizontalni
smeri ločil vektor $\Delta u$ in v vertikalni $\Delta v$. Prikazane so na sliki \ref{fig:oznake}.

\begin{figure}[h]
	\centering
	\includegraphics[width=\textwidth]{vidno_okno}
	\caption{Oznake prikazane vizualno}
	\label{fig:oznake}
\end{figure}

Potem lahko žarek $\vec{r}$, ki potuje skozi piksel $(i,j)$, parametriziramo s $t$:

\begin{equation}
	\vec{r} = C + t \cdot (P_{00} + i \cdot \Delta u + j \cdot \Delta v - C)
\end{equation}

V prihodnje pa bomo obravnavali splošne žarke z izvorom v $O$ in smerjo $\vec{d}$.

\section{Detekcija trka}

Trenutno naši žarki samo streljajo v praznino, zato bomo zdaj v svet dodali objekte v katere lahko žarki trčijo.
Matematično je najlažje opisati sfero. Splošna enačba za sfero s središčem $C$ in radijem $r$ je

\begin{equation}
	\label{eq:sfera}
	(C_{x} - x)^2  + (C_{y} - y)^2  + (C_{z} - z)^2  = r^2
\end{equation}

Če točko $(x,y,z)$ označimo kot $P$ lahko enačbo \ref{eq:sfera} zapišemo kot skalarni produkt.

\begin{equation}
	(C - P)(C - P) = r^2
\end{equation}

Naša točka pa leži na žarku, torej dobimo:

\begin{equation}
	(C - (O + t \vec{d}))(C - (O + t \vec{d})) = r^2
\end{equation}

Po nekaj preureditvah enačbe dobimo naslednjo kvadratno enačbo

\begin{equation}
	t^2 \lVert C \rVert ^2  - 2 t \cdot \vec{d} (C - O) + \lVert C - O \rVert ^2  - r^2 = 0
\end{equation}

S pomočjo diskriminante lahko zlahka preverimo ali naš žarek sploh zadane sfero. Če obstajata dve rešitvi, vzamemo
najmanjšo pozitivno (torej najbližjo kameri v smeri pogleda). V primeru, da pozitivna rešitev ne obstaja pa
smatramo, kot da žarek ni zadel objekta, saj je za kamero.

Na sliki \ref{fig:red_sphere} vidimo primer izrisa sfere, kjer jo obarvamo z rdečo barvo.

\begin{figure}[H]
	\label{fig:red_sphere}
	\includegraphics[width=\textwidth]{red_sphere}
	\caption{Rdeče obarvana sfera}
\end{figure}

\subsection{Antialiasing}

Na zgornji sliki vidimo, da ima sfera žagast rob zaradi diskretizacije pikslov v mrežo. Temu
pojavimo pravimo \textit{aliasing}. Znebimo se ga tako, da za vsak piksel izstrelimo več žarkov, ki so naključno
pertrubirani za največ polovično razdaljo med piksli, ter nato barvo teh žarkov povprečimo. Tehnika se imenuje
MSAA (\textit{multisample anti-aliasing}) in smo jo izbrali, ker je v kontekstu sledenja žarkov najbolj
enostavna za implementirati.

\section{Materiali}

\subsection{Difuzni materiali in Lambertov odboj}

Prvi materiali, ki se jih bomo lotili, so difuzni. V realnem svetu zgledajo mat, ker se žarki naključno odbijejo od
površine objekta. Lambertov odboj je aproksimacija difuznih materialov, ki pravi, da naj je smer odbitega žarka
izbrana sorazmerno s količino $\cos (\phi) $, kjer je $\phi$ vpadni kot.

\begin{figure}[H]
	\centering
	\includegraphics[height=250pt]{shema_difzuni}
	\caption{Lambertov odboj}
\end{figure}

Če žarek seka objekt na točki $P$ in je normala površine vektor $n$, potem izberemo naključno točko $Q$
znotraj enotske sfere s središčem $P + n$ (kjer je normala enotski vektor). Smer odbitega žarka je potem
$Q - P$.

\begin{figure}[H]
	\centering
	\includegraphics[width=400pt]{difuzni}
	\caption{Dve difuzni sferi}
\end{figure}

Da dodamo barvo, odbiti žarek pomnožimo z barvo objekta, ki ji pravimo \textit{albedo}. Tako se barva
svetlobnega vira postopoma spreminja ob vsakem odboju - množi se z albedom vsakega objekta, s katerim se
žarek sreča. Končna barva piksla je zato produkt barve svetlobnega vira in vseh albedo vrednosti
objektov na poti žarka.

\textit{Opomba: }, ker se ta metoda zanaša na naključne procese za aproksimacijo realnega sveta (Monte Carlo),
dobimo nekaj šuma v sliki.

\subsection{Kovinski materiali}

Kovine imajo svojo barvo, vendar pa hkrati tudi odbijejo del svetlobe pod istim kotom, kot je
žarek srečal objekt. Odbiti žarek je sestavljen iz dveh delov: zrcaljeni del in t.i. zamegljeni del
(\textit{fuzziness}). Končni odbiti žarek je vsota teh dveh vektorjev.

\begin{figure}[H]
	\centering
	\includegraphics[width=400pt]{metal}
	\caption{Odboj žarka od kovinskega materiala}
\end{figure}

Zrcaljeni del je preprosto vpadni žarek zrcaljen čez ploskev, ki jo opiše normala in z zamenjano smerjo.
Zameglejni del pa simulira hrapavost kovine z naključnim vektor znotraj enotske sfere pomnoženim
s parametrom $0 \leq f \leq 1$, ki kontrolira kako zamegljen je objekt.

\subsection{Dielektrični materiali}
Še zadnja vrsta materialov, ki jih uporabljamo, so prosojni materiali oziroma dielektriki. To so na primer voda,
steklo in diamant. Ko svetloba prehaja med dvema materialoma z različnima lomnima količnikoma, se upogne.
Na primer, zrak ima količnik blizu $1$, steklo pa okoli $1.5$. Lom svetlobe opisujemo z lomnim zakonom, ki nam
pove razmerje vpadnim kotom $\theta$ in kotom lomljenega žarka $\theta'$. Lomne količnike materialov pa označimo
z $\eta$ in $\eta'$. S pomočjo tega zakona lahko izrazimo lomljeni žarek $\vec{t}$, ki preide skozi mejo med materialoma
z normalo $\vec{n}$.

\begin{equation}
	\begin{aligned}
		\eta \sin\theta & = \eta' \sin\theta',                                                              \\
		\cos\theta      & = -\frac{\vec{d}}{\lVert \vec{d} \rVert} \cdot \vec{n},                           \\
		\vec{t}         & = \eta_1 \left(\frac{\vec{d}}{\lVert \vec{d} \rVert} + \cos\theta\,\vec{n}\right)
		- \vec{n}\,\sqrt{1 - \eta_{1}^2\!\left(1 - \cos^2\theta\right)}, \quad \eta_{1} = \frac{\eta}{\eta'}.
	\end{aligned}
\end{equation}

\begin{figure}[H]
	\centering
	\includegraphics[width=\textwidth]{lomni_zakon}
	\caption{Lomni zakon}
\end{figure}

Lomni zakon ni vedno rešljiv — pri prevelikem vpadnem kotu nastopi popolni notranji odboj — svetloba se v takem
primeru v celoti odbije od meje med materialoma, namesto da bi prešla v drugi medij. Tudi če je enačba rešljiva,
se svetloba ne lomi v celoti, nek delež se vedno odbije od materiala. Točen delež opisujejo Fresnelove enačbe, ki
so odvisne od vpadnega kota. Ker pa je reševanje tega sistema počasno, v praksi uporabimo polinomsko Schlickovo
aproksimacijo, ki nam pove delež odbite svetlobe.

\begin{equation}
	\begin{aligned}
		R(\theta) & \approx R_0 + (1 - R_0)\,(1 - \cos\theta)^5,  \\
		R_0       & = \left(\frac{n_1 - n_2}{n_1 + n_2}\right)^2.
	\end{aligned}
\end{equation}

Torej v praksi bomo uporabljali Monte Carlo metodo, kjer se žarek odbije od površine materiala, če lomni zakon
ni rešljiv ali pa velja $\verb|naključno število| < \verb|Schlickova aproksimacija|$ (torej vzorcimo
iz Bernoullijeve distribucije z verjetnostjo $R(\theta)$).

\section{Globinska ostrina}

V pravi kameri je leča sestavljena iz več steklenih elementov, ki so za odprtino, ki ji pravimo zaslonka.
Zaradi tega, ker je ta odprtina večja od točke, bodo objekti na določeni razdalje povsem ostri, ostali pa
zamegljeni. Temu pojavu pravimo globinska ostrina. Namesto modeliranja kompleksnega sistema večih elementov
pa bomo uporabili model tanke leče.

% skica tanke lece in zakaj zarki niso ostri
\begin{figure}[H]
	\centering
	\includegraphics[width=\textwidth]{leca}
	\caption{Shema tanke leče}
\end{figure}

Efekt bomo simulirali tako, da se bomo pretvarjali, da je pozicija naše kamere v resnici pozicija leče in
da je položaj našega vidnega okna ravnina, kjer bodo objekti izostreni.

Kako to dosežemo? Izhodišča žarkov naključno zamaknemo za vektor, izbran v enotskem krogu, in ga skaliramo s
faktorjem, izračunanim glede na odprtost zaslonke. Ti žarki so nato usmerjeni točno v ciljni piksel na vidnem
oknu, vendar zaradi zamaknjenih izhodišč malo odstopajo od pričakovane poti pred in po oknu, kar ustvari učinek
globinske ostrine.

\printbibliography

\end{document}
